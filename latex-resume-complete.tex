% resume.tex
% vim:set ft=tex spell:

\documentclass[10pt,letterpaper]{article}
\usepackage[letterpaper,margin=0.75in]{geometry}
\usepackage[utf8]{inputenc}
\usepackage{mdwlist}
\usepackage[T1]{fontenc}
\usepackage{textcomp}
\usepackage{tgpagella}
\usepackage{ifthen}
\pagestyle{empty}
\setlength{\tabcolsep}{0em}

% indentsection style, used for sections that aren't already in lists
% that need indentation to the level of all text in the document
\newenvironment{indentsection}[1]%
{\begin{list}{}%
	{\setlength{\leftmargin}{#1}}%
	\item[]%
}
{\end{list}}

% opposite of above; bump a section back toward the left margin
\newenvironment{unindentsection}[1]%
{\begin{list}{}%
	{\setlength{\leftmargin}{-0.5#1}}%
	\item[]%
}
{\end{list}}

% format two pieces of text, one left aligned and one right aligned
\newcommand{\headerrow}[2]
{\begin{tabular*}{\linewidth}{l@{\extracolsep{\fill}}r}
	#1 &
	#2 \\
\end{tabular*}}

\newcommand{\basicitem}[5]
{
	\item
	\headerrow
		{\textbf{#1}}
		{\textbf{#2}}
	\ifthenelse{\equal{#3}{} \AND \equal{#4}{}}
		{}
		{
			\headerrow
			{\emph{#3}}
			{\emph{#4}}
		}
	#5
}

% make "C++" look pretty when used in text by touching up the plus signs
\newcommand{\CPP}
{C\nolinebreak[4]\hspace{-.05em}\raisebox{.22ex}{\footnotesize\bf ++}}

% and the actual content starts here
\begin{document}

\begin{center}
{\LARGE \textbf{Noah E. Wolfe}}

newolfe@ncsu.edu\ \ \textbullet
\ \ (704) 998-1322
\\
2201 Dunn Avenue\ \ \textbullet
\ \ Bagwell\ 115\ \ \textbullet
\ \ Raleigh, NC 27607

\end{center}

%%% Research/Projects
\hrule
\vspace{-0.4em}
\subsection*{Research/Projects}

\begin{itemize}
	\parskip=0.1em

	\basicitem
		{Optical Simulation for Tandem Organic Photovoltaics}
		{August 2018 - Present}
		{https://adegroup.wordpress.ncsu.edu/}
		{}
		{Writing an optical simulation with Python to determine the optimum thickness of organic photoreactive layers and recombination layers in a tandem organic photovoltaic device, for use in both the Ade Research Group, and other research groups within the NCSU Organic and Carbon Electronics Lab (ORaCEL). These determined characteristics will then be used to build and test organic photovoltaic devices, in an attempt to replicate an efficiency of 17\% reported in prior tandem organic photovoltaics research. The primary challenges of this project have been optimization of the code so that it can effectively run on everyday computers and laptops, the identification of reasons my simulation may produce results which differ from similar simulation codes, and adaptation to a field of physics which I had no prior exposure to.}
		
	\basicitem
		{Low-Cost Mesh Networks to Monitor Air Quality}
		{June 2017 - Present}
		{https://github.com/noahewolfe/pollution-sensor-mesh-node}
		{}
		{Designing and building low-cost mesh networks, to solve issues of intra-urban air pollutant gradient monitoring. Typically, air quality monitoring is done through central-site ambient monitors, which can be lacking in the developing world, and even some American cities. Since recent research has shown that intra-urban pollutant gradients can have a significant affect on respiratory health, my aim is to create a mesh network of pollutant monitors which can be easily deployed in a city, at low cost, to fill these monitoring gaps. This project has involved a number of skills/technologies, including software and hardware development for Arduino Uno and Raspberry Pi, development of a basic data compression protocol, and use of LoRa packet radio. }
		
	\basicitem
		{Open Star Cluster Simulation}
		{June 2015 - Present}
		{https://github.com/noahewolfe/clusters}
		{}
		{Writing a simulation application, written in Python and using the AMUSE framework. This is a simulation of open star clusters, based upon a Plummer model and Salpeter mass distribution, demonstrating and exploring the link between open cluster collisions and the emergence of Blue Straggler Stars, such as in R136. The primary technical challenge has been optimization of the code to run on my personal laptop, while I have also been academically challenged, and very excited, as I learn new astrophysics concepts.}
		
	\basicitem
		{Custom-Built Wide Band Radio Telescope}
		{March 2017 - August 2017}
		{https://goo.gl/R1gKfy}
		{}
		{Prototyped a small (~0.5 m) radio telescope, using a unique combination of cutting-edge software defined radio equipment and a repurposed satellite TV dish, attempting to detect electromagnetic phenomena (whistlers) in Jupiter’s magnetosphere.}
		
	\basicitem
		{Meeting Magic}
		{February 2016 - June 2016}
		{https://github.com/thezenth/Meeting-Magic}
		{}
		{Created a Node.js powered app which plans meetings for multiple users based upon “big data,” such as local traffic and weather, as well as personal data such as each user’s scheduled trips or food preferences. This application was awarded the highest honors in the high school division of the ISS Inaugural Coding Competition.}

\end{itemize}

%%% Community Service
\hrule
\vspace{-0.4em}
\subsection*{Community Service}
\begin{itemize}
	\parskip=0.1em

	\basicitem
		{Mu Alpha Theta Peer Tutoring}
		{August 2015 - June 2018}
		{}
		{}
		{Actively tutored students, both in the classroom, but especially in one-on-one peer tutoring, focusing on the Math 2 / Math 3 / Precalculus levels of mathematics. The majority of my tutees were students who struggled to balance extracurricular priorities with their core mathematics education; every tutee not only eventually learned new mathematics skills, but also new balance and organizational skills as well.}

	\basicitem
		{Tour Application for PARI}
		{August 2016 - Present}
		{http://www.pari.edu}
		{}
		{Currently leading the creation of a smartphone tour app as a volunteer, for the Pisgah Astronomical Research Institute, powered by Node.js and PostgreSQL. This not only taught me new technical skills, from how to use PostgreSQL to Department of Defense security standards (as PARI used to be a DoD facility), but I have also learned how to manage a project and coordinate with multiple people, even at a long distance.}

	\basicitem
		{Coder Dojo}
		{October 2017 - June 2018}
		{}
		{}
		{Volunteered at the “CoderDojo,” a club at a local middle school designed to introduce young students to programming at all levels. I taught Python to the more advanced students, guiding them in the creation of a text-based adventure game, while also working on introducing other students to the basics of programming with technology like the Raspberry Pi and Scratch.}

	\basicitem
		{People Enjoying People (PEP) Camp}
		{August 2017}
		{}
		{}
		{Volunteer at a week-long summer camp for students with various disabilities; each day, I would be paired with a different student, including students with Autism and Cerebral Palsy, playing with them in different ways and encouraging them to play with other students as well.}

	\basicitem
		{Habitat for Humanity}
		{September 2018 - Present}
		{}
		{}
		{}
	
\end{itemize}

%%% Leadership and Extracurriculars
\hrule
\vspace{-0.4em}
\subsection*{Leadership and Extracurriculars}
\begin{itemize}
	\parskip=0.1em

	\basicitem
		{Science Club}
		{September 2016 - June 2018}
		{President and Co-Founder}
		{September 2016 - June 2018}
		{Club whose goal is to foster a community of scientists, thinkers, and learners. This is accomplished in three primary manners; through active, engaged discussion of modern topics and debates in science, competitions, experiments, and other interactive activities, and tutoring to give back to the high school community through science.}

	\basicitem
		{Mu Alpha Theta}
		{August 2015 - June 2018}
		{Vice President}
		{August 2017 - June 2018}
		{As Vice President, my duties center around management of tutoring programs, including updating existing tutoring infrastructure with technologies such as the Google Apps Script and Google Sheets. I also participate in the rest of the club activities, including math competitions, and peer and classroom tutoring.}

	\basicitem
		{Marching Band Front Ensemble}
		{August 2014 - June 2017}
		{Second-most senior member, Marimba}
		{August 2015 - June 2017}
		{Marching band, and the band program in general, has not only helped to unlock an everlasting love for music and the arts, but it has unlocked leadership and personal focus skills which I had struggled with previously.}

	\basicitem
		{Park Scholars Class of 2022 Legacy Committee}
		{September 2018 - Present}
		{Co-Chair}
		{September 2018 - Present}
		{The goal of this committee is to define a class legacy for the Park Scholars Class of 2022. As co-chair, I have guided this interdiscplinary committe in the definition of a vision for the class legacy, without any previous guidelines or requirements for the legacy. Our goal is to harness the wide-ranging interests and expertise of the Park Class of 2022 to make a positive impact on the NC State, North Carolina, or even broader community.}

	\basicitem
		{Astronomy Club}
		{August 2018 - Present}
		{}
		{}
		{}

	\basicitem
		{Society of Physics Students}
		{August 2018 - Present}
		{}
		{}
		{}


\end{itemize}

%%% Academic Awards and Honors

\hrule
\vspace{-0.4em}
\subsection*{Awards and Honors}

\begin{itemize}
	\parskip=0.1em

	\basicitem
		{Outstanding Percussion Award, Concert Band}
		{May 2015}
		{}
		{}
		{}

	\basicitem
		{Most Improved Percussion Award, Marching Band}
		{May 2015}
		{}
		{}
		{}

	\basicitem
		{Highest Honors, High-School Division, ISS Inaugural Coding Competition}
		{May 2016}
		{}
		{}
		{}

	\basicitem
		{Daughters of the American Revolution Good Citizen Award}
		{May 2018}
		{}
		{}
		{}

	\basicitem
		{University Scholars Program}
		{August 2018 - Present}
		{https://scholars.dasa.ncsu.edu}
		{}
		{The University Scholars Program (USP) at NCSU exposes students to a diverse experiences and perspectives. Through this program, I have engaged in activities including: learned basic orienteering at Raleigh’s Lake Crabtree, critically discussed the Frontline documentary “Left Behind Amierca”, and listened to a talk by a National Geographic photojournalist. }

	\basicitem
		{Park Scholarships}
		{August 2018 - Present}
		{https://park.ncsu.edu/}
		{}
		{NC State University's Park Scholarship is a highly selective, full merit scholarship awarded on the basis of outstanding accomplishments and potential in scholarship, leadership, service, and character. As a Park Scholar, I am participating in a four-year, executive-style leadership academy; diversity training; a year-long civic engagement project; and intensive learning laboratories exploring leadership challenges regionally and nationally.}

\end{itemize}


%%% Education

\hrule
\vspace{-0.4em}
\subsection*{Education}

\begin{itemize}
	\parskip=0.1em

	\basicitem
		{North Carolina State University}
		{Raleigh, NC}
		{B.S. Physics, GPA: 4.146}
		{August 2018 - May 2022 (expected)}
		{}

	\begin{description*}
		\item[Relevant Coursework:]
		University Physics I, Special Topics in Physics, Calculus III, University Physics II (Ongoing), Introduction to Scientific Computing (Ongoing), Applied Differential Equations I (Ongoing), Linear Algebra (Ongoing)
	\end{description*}

\end{itemize}

%%% Skills

\hrule
\vspace{-0.4em}
\subsection*{Skills}

\begin{indentsection}{\parindent}
\hyphenpenalty=1000
\begin{description*}
	\item[Programming:]
	Python (Advanced), C and \CPP (Fundamental), Java (Intermediate), general Unix/Linux proficiency (Ubuntu, CentOS, Raspbian), Mathematica (Intermediate), MatLab (Fundamental), MongoDB (Fundamental), PostgreSQL (Intermediate), HTML and template engines incl. ejs (Advanced), JavaScript and node.js (Advanced)
	\item[Other Technical Experience:]
	Soldering/electronics (Intermediate), Arduino Uno, Raspberry Pi, \LaTeX (Intermediate)
	\item[Languages:]
	English (native), Gujarati (Fundamental), Spanish (Fundamental)
\end{description*}
\end{indentsection}

\end{document}
