% resume.tex
% vim:set ft=tex spell:

\documentclass[10pt,letterpaper]{article}
\usepackage[letterpaper,margin=0.75in]{geometry}
\usepackage[utf8]{inputenc}
\usepackage{mdwlist}
\usepackage[T1]{fontenc}
\usepackage{textcomp}
\usepackage{tgpagella}
\usepackage{ifthen}
\pagestyle{empty}
\setlength{\tabcolsep}{0em}

% indentsection style, used for sections that aren't already in lists
% that need indentation to the level of all text in the document
\newenvironment{indentsection}[1]%
{\begin{list}{}%
	{\setlength{\leftmargin}{#1}}%
	\item[]%
}
{\end{list}}

% opposite of above; bump a section back toward the left margin
\newenvironment{unindentsection}[1]%
{\begin{list}{}%
	{\setlength{\leftmargin}{-0.5#1}}%
	\item[]%
}
{\end{list}}

% format two pieces of text, one left aligned and one right aligned
\newcommand{\headerrow}[2]
{\begin{tabular*}{\linewidth}{l@{\extracolsep{\fill}}r}
	#1 &
	#2 \\
\end{tabular*}}

\newcommand{\basicitem}[5]
{
	\item
	\headerrow
		{\textbf{#1}}
		{\textbf{#2}}
	\ifthenelse{\equal{#3}{} \AND \equal{#4}{}}
		{}
		{
			\headerrow
			{\emph{#3}}
			{\emph{#4}}
		}
	#5
}

% make "C++" look pretty when used in text by touching up the plus signs
\newcommand{\CPP}
{C\nolinebreak[4]\hspace{-.05em}\raisebox{.22ex}{\footnotesize\bf ++}}

% and the actual content starts here
\begin{document}

\begin{center}
{\LARGE \textbf{Noah E. Wolfe}}

2201 Dunn Avenue\ \ \textbullet
\ \ Bagwell\ 115\ \ \textbullet
\ \ Raleigh, NC 27607
\\
(704) 998-1322\ \ \textbullet
\ \ newolfe@ncsu.edu
\end{center}

%%% Research/Projects
\hrule
\vspace{-0.4em}
\subsection*{Research/Projects}

\begin{itemize}
	\parskip=0.1em

	\basicitem
		{Custom-Built Wide Band Radio Telescope}
		{March 2017 - August 2017}
		{Complete}
		{https://goo.gl/R1gKfy}
		{Prototyped a small (~0.5 m) radio telescope, using a unique combination of cutting-edge software defined radio equipment and a repurposed satellite TV dish, attempting to detect electromagnetic phenomena (whistlers) in Jupiter’s magnetosphere.}

	\basicitem
		{Meeting Magic}
		{February 2016 - June 2016}
		{Complete}
		{https://github.com/thezenth/Meeting-Magic}
		{Created a Node.js powered app which plans meetings for multiple users based upon “big data,” such as local traffic and weather, as well as personal data such as each user’s scheduled trips or food preferences. This application was awarded the highest honors in the high school division of the ISS Inaugural Coding Competition.}

	\basicitem
		{Open Star Cluster Simulation}
		{June 2015 - Present}
		{Ongoing}
		{https://github.com/thezenth/Cluster-Collision}
		{Designed and built a simulation application, written in Python and using the AMUSE framework. This was a simulation of open star clusters, based upon a Plummer model and Salpeter mass distribution, demonstrating and exploring the link between open cluster collisions and the emergence of Blue Straggler Stars, such as in R136.}

	\basicitem
		{Optical Simulation for Tandem Organic Photovoltaics}
		{August 2018 - Present}
		{Ongoing}
		{https://tinyurl.com/ya8x3lxy}
		{Writing an optical simulation with Python to determine the optimum thickness of organic photoreactive layers and recombination layers in a tandem organic photovoltaic device, for use in both the Ade Research Group, and other research groups within the NCSU Organic and Carbon Electronics Lab (ORaCEL). These determined characteristics will then be used to build and test organic photovoltaic devices.}

\end{itemize}

%%% Community Service
\hrule
\vspace{-0.4em}
\subsection*{Community Service}
\begin{itemize}
	\parskip=0.1em

	\basicitem
		{Tour Application for PARI}
		{http://www.pari.edu}
		{}
		{August 2017 - Present}
		{Currently leading the creation of a smartphone tour app as a volunteer, for the Pisgah Astronomical Research Institute, powered by Node.js and PostgreSQL. This not only taught me new technical skills, from how to use PostgreSQL to Department of Defense security standards (as PARI used to be a DoD facility), but I have also learned how to manage a project and coordinate with multiple people, even at a long distance.}

	\basicitem
		{Mu Alpha Theta Peer Tutoring}
		{August 2015 - Present}
		{}
		{}
		{Actively tutored students, both in the classroom, but especially in one-on-one peer tutoring, focusing on the Math 2 / Math 3 / Precalculus levels of mathematics. The majority of my tutees were students who struggled to balance extracurricular priorities with their core mathematics education; every tutee not only eventually learned new mathematics skills, but also new balance and organizational skills as well.}

\end{itemize}

\hrule
\vspace{-0.4em}
\subsection*{Education}

\begin{itemize}
	\parskip=0.1em

	\item 
	\headerrow
		{\textbf{Somewhere State University}}
		{\textbf{Somewhere, SM}}
	\\
	\headerrow
		{\emph{College of Engineering, B.S. Computer Science}}
		{\emph{1995 -- 1999}}
	\begin{itemize*}
		\item Lorem ipsum dolor sit amet, consectetuer adipiscing elit.
		\item Mirum est notare quam littera gothica, quam nunc putamus parum
		claram.
	\end{itemize*}

\end{itemize}


\hrule
\vspace{-0.4em}
\subsection*{Skills}

\begin{indentsection}{\parindent}
\hyphenpenalty=1000
\begin{description*}
	\item[Technical:]
	Python (Advanced), C and \CPP (Fundamental), Java (Intermediate), JavaScript (Advanced), \LaTeX (Intermediate), PostgreSQL (Intermediate), general Unix proficiency
	\item[Languages:]
	English (native), Gujarati (Fundamental), Spanish (Fundamental)
\end{description*}
\end{indentsection}

\end{document}
